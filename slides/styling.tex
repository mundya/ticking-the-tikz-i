\begin{frame}{The Need for Styling}
	Often have many
	\begin{itemize}
		\item similar elements in diagram (e.g.\ vectors, guidelines)
		\item classes of diagram in document (e.g.\ Petri nets, FSM)
	\end{itemize}
	Don't want to change every one individually!

	Applying styles is good practise.
\end{frame}

\begin{frame}{The Need for Styling}
	In the previous example
	\begin{itemize}
		\item One vector
		\item Two guidelines
	\end{itemize}
	Could have many more -- styles make this easy
\end{frame}

\begin{frame}{Defining Styles}
	Styles define a set of options with one name, created using:
	\begin{description}
		\item[\texttt{\textbackslash tikzstyle}] Easier to use, less useful
		\item[\texttt{\textbackslash tikzset}] Preferred method
	\end{description}

	\texttt{[very thick, blue, ->]} becomes \texttt{[vector]} after
	\lstinputlisting{figures/styling/1}

\end{frame}

\begin{frame}[plain]{Back to the Example}

\end{frame}
